% Simple article, but with some useful ApJ things included
% Created by AYQH, 18 June 2016
\documentclass[12pt, letterpaper, preprint]{aastex}
\bibliographystyle{apj}
\usepackage{float, bm, graphicx, subfigure, amsmath, morefloats}
\usepackage{color}
\usepackage{tabularx}

% useful macros
\newcommand{\angstrom}{\mbox{\AA}}

\begin{document}

\title{2015 ``Will Ask'' Qual Questions}

\section*{Radiative Processes}

\begin{enumerate}

  \item \textbf{Derive the total power and characteristic frequency 
      of synchrotron radiation from a relativistic particle of mass $m$, 
      charge $e$, and energy $E$ moving in a magnetic field $B$. 
      Use this to explain why synchrotron radiation is generally 
      negligible for protons.}

\textbf{Synchrotron radiation} is the energy radiated by a relativistic 
charged particle as it moves in a magnetic field.

Let's consider the case of an electron moving in a uniform 
external magnetic field.
The electron experiences a Lorentz force:

\begin{equation}
  \textbf{F} = e \, (\frac{\mathbf{v}}{c} \times \mathbf{B})
  \label{lorentz-force}
\end{equation}

The equation of motion of the particle can thus be written

\begin{equation}
  \textbf{F} = \frac{d \mathbf{p}}{dt} = \frac{d (\gamma m \mathbf{v})}{dt}
  = \gamma m \frac{d \mathbf{v}}{dt} 
  = e \, (\frac{\mathbf{v}}{c} \times \mathbf{B})
  \label{eom}
\end{equation}

Notice that the electron's speed does not change, because
the magnetic field can only act perpendicular to its motion.
Thus, only the direction of the electron can change.
From this, you can see that both the parallel and perpendicular
components of the velocity (wrt the magnetic field) remain constant.

The net result is that you have a constant magnetic field in which
an electron is moving along a field line on a uniform helical path
(or with zero velocity along a field line, a circular path)
with a constant linear and angular speed. 
There is no force along a field line. 

The quantity $\frac{d\mathbf{v}}{dt}$ is acceleration.
Solve for it:

\begin{equation}
  \mathbf{a} = \frac{e (\mathbf{v} \times \mathbf{B})}{\gamma m c}
  \label{<++>}
\end{equation}

And in the instantaneous inertial frame of the electron,
the velocity is perpendicular to the magnetic field and
we can regard the acceleration as the centripetal acceleration.
$\frac{v^2}{r}$. Setting $v = r \omega$, we get a
characteristic frequency at which the electron is
spiralling, called the \textbf{gyrofrequency}:

\begin{equation}
  \omega_g = \frac{eB}{\gamma m_e c}
  \label{gyrofrequency}
\end{equation}

This is, however, not the characteristic frequency
that we observe. We must take into account a number
of relativistic effects, including that power must
be multiplied by $\gamma^2$, a combination of
relativistic beaming and Doppler effect that turns
a sinusoidal light curve into a series of sharp pulses.
The resulting radiation spectrum peaks at $\gamma^3 \omega_g$
which is three times the gyrofrequency.

The \textbf{relativistic Larmor formula} gives the power radiated
from a single relativistic electron:

\begin{equation}
  P = \frac{2}{3} \frac{\gamma^4 e^2 a^2}{c^3}
  \label{larmor-formula}
\end{equation}

You can derive it via Lorentz transformations,
taking the non-relativistic Larmor formula to hold
true in the frame of the particle.

Using our expression for acceleration above,
we get that the power radiated is

\begin{equation}
  P = \frac{4}{3} \, \sigma_T \, c \, \gamma^2 \, \beta^2 \, u_B
  \label{relativistic-larmor}
\end{equation}

The synchrotron radiation loss for a particle scales
like its gamma. So the loss rate for an electron is
over $10^{13}$ times the loss for a proton of the same energy.
(I think that it's just because of the mass?)

  \item \textbf{Explain the connection between detailed balance, 
    the Einstein A and B relations,
    Kirchoff’s law and the Milne relations, 
    and give an example of their use to connect the
    bremsstrahlung emission spectrum and the free-free absorption coefficient.}

The Einstein relations link together atomic properties and
thus macroscopic conditions (like thermal equilibrium) don't matter.
Consider a two-level system, the top level characterized by
$L_2, g_2$ and the bottom level characterized by $L_1, g_1$. 
The energy gap is $h\nu$. The different Einstein coefficients
characterize the transition probability per unit time for
three different mechanisms: $A_{21}$ for spontaneous emission
(no radiation field required), $B_{12} \bar{J}$ for absorption,
and $B_{21} \bar{J}$ for stimulated emission.

\begin{equation}
  \bar{J} \equiv \int_0^\infty J_\nu \phi(\nu) d\nu
  \label{jbar}
\end{equation}

Note that the B's are both proportional to the mean intensity
of the radiation field.

The detailed balance equation says that the number of transitions into level 2
equals the number of transitions out of level 2. In other words,
in thermal equilibrium, the number of transitions into an atomic state
equals the number of transitions out of the state. 
It can be written as follows:

\begin{equation}
  n_1 B_{12} \bar{J} = n_2 B_{21} \bar{J} + n_2 A_{21}
  \label{detailed-balance}
\end{equation}

The Boltzmann Equation governing thermal equilibrium is

\begin{equation}
  \frac{n_1}{n_2} = \frac{g_1}{g_2} e^{h \nu / k_B T}
  \label{boltzmann-eq}
\end{equation}

Using this to solve for $\bar{J}$ we get

$$ \bar{J} = \frac{A_{21} / B_{21}}{(g_1 B_{12}/g_2 B_{21}) e^{h\nu/kT} - 1} $$

This is equal to 

$$ B_\nu = \frac{2 h \nu^3}{c^2} \frac{1}{e^{h\nu/kT}-1} $$

This gives the Einstein relations:

\begin{equation}
  \frac{A_{21}}{B_{21}} = \frac{2 h \nu^3}{c^2}
  \label{einstein-A}
\end{equation}

\begin{equation}
  g_1 B_{12} = g_2 B_{21}
  \label{einstein-B}
\end{equation}

Notice that the temperature dependence drops,
so the Einstein relations hold even outside of
thermal equilibrium.

\textbf{Kirchoff's law for thermal equilibrium}
says that a good emitter is a good absorber, in
thermal equilibrium. The moral of the story is that
there is clearly a relationship between emission
and absorption at a microscopic level.

\begin{equation}
  j_\nu = \alpha_\nu B_\nu (T)
  \label{kirchoff-law}
\end{equation}

The \textbf{Milne Relation} governs the inverse process
of photoionization, that is an ion recapturing an electron
and emitting a photon. Let $\sigma_{fb}$ be the cross-section
for recapture. We assume complete thermal equilibrium
and derive a result that will end up being independent of
thermal equilibrium. 

Thermal equilibrium dictates that the rate of radiative
recombinations must equal the rate of photoionization.

\textbf{Bremsstrahlung}, or \textbf{free-free emission},
involves an electron whizzing by an ion. There is radiation
emitted due to the acceleration of charge in the Coulomb field
of another charge.

Remember, a good emitter is a good absorber. Using Kirchoff's Law,

$$ j_\nu^{ff} = \alpha_\nu^{ff} B_\nu (T) $$

\begin{equation}
  \frac{dW}{dt dV dv} = \epsilon_\nu^{ff} = 4 \pi j_\nu^{ff} 
  = 6.8 \times 10^{-38} T^{-1/2} Z^2 n_e n_i e^{h \nu / kT}
  \bar{g}^{ff} (\nu,T)
  \label{bremsstrahlung}
\end{equation}

But you also have Kirchhoff's Law (thermal); both of these
processes are happening because we're in thermal equilibrium.
$S_\nu = B_\nu$, and $j_\nu = \alpha_\nu B_\nu$. 

\begin{equation}
  \alpha_\nu^{ff} = 3.7 \times 10^8 T^{-1/2} Z^2 n_e n_i \nu^{-3}
  (1-e^{-h\nu/kT}) g^{ff}
\end{equation}

In the Wien limit, you get $\alpha_\nu^{ff} \propto \nu^{-3} T_{-1/2}$.
In the RJ limit, you get $\alpha_\nu^{ff} \propto \nu^{-2} T^{-3/2}$.

So, you preferantially absorb low-energy electrons.


  \item Draw the energy levels of the hydrogen atom and 
    identify which transitions are allowed. 
    Which ones are in the visible part of the spectrum? 
    Which level has no allowed decays, and what is its main decay mode?
\end{enumerate}

\section*{Instrumentation}

\begin{enumerate}
  \item \textbf{Describe quantitatively the point spread function 
      of a diffraction-limited optical telescope. 
      Explain how diffraction spikes arise, 
      and what determines their position and intensities. 
      Under what circumstances will the PSF be broadened by atmospheric
      turbulence?}
\end{enumerate}

\section*{Stars}
\begin{enumerate}
  \item \textbf{Explain what the Hayashi track is, 
      and describe what types of objects live on it.
      Qualitatively explain how it arises and what 
      assumptions are required for its derivation.}
  \item \textbf{Estimate the temperature as a function 
      of depth in the Sun’s convection zone.
      What is the temperature at the base of the convection zone?}
\end{enumerate}

\end{document}


