% Simple article, but with some useful ApJ things included
% Created by AYQH, 18 June 2016
\documentclass[12pt, letterpaper, preprint]{aastex}
\bibliographystyle{apj}
\usepackage{float, bm, graphicx, subfigure, amsmath, morefloats}
\usepackage{color}
\usepackage{tabularx}

\setlength\parindent{0pt}

% useful macros
\newcommand{\ledd}{L_{\mathrm{Edd}}}
\newcommand{\medd}{\dot{M}_{\mathrm{Edd}}}
\newcommand{\Comment}[2]{ [{\color{red}\sc #1 :} {{\color{cyan} \it #2}}]}
\newcommand{\gcm}{g/$\mathrm{cm}^3$}
\newcommand{\angstrom}{\mbox{\AA}}
\newcommand{\vel}{\textbf{v}}
\newcommand{\bfield}{\textbf{B}}
\newcommand{\acc}{\textbf{a}}

\begin{document}

\title{Miscellaneous Notes}

\section*{Radiative Processes}

There is a distinction between \emph{thermal radiation}
and \emph{blackbody radiation}.
The term \emph{thermal radiation} is a statement about the
matter emitting the radiation: that the matter is in thermal
equilibrium (with its surroundings).
$S_\nu = B_\nu$.
The term \emph{blackbody radiation} is a statement about the
radiation itself: that it itself is in thermal equilibrium.
It is the limit of thermal radiation for optically thick media.
In this limit, $I_\nu = B_\nu$.

\section*{Instrumentation}

\section*{Stars}
\begin{enumerate}
  \item \textbf{Explain what the Hayashi track is, 
      and describe what types of objects live on it.
      Qualitatively explain how it arises and what 
      assumptions are required for its derivation.}
  \item \textbf{Estimate the temperature as a function 
      of depth in the Sun’s convection zone.
      What is the temperature at the base of the convection zone?}
\end{enumerate}

\section*{Galaxies}

\begin{enumerate}

\item \textbf{Define two-body relaxation and estimate its time scale in (a) a globular cluster, (b) the Milky Way's disk. In which cases is two-body relaxation important?}

\item \textbf{Draw qualitatively the spectral energy distribution of the Milky Way, and describe how its morphology might appear to an external observer as a function of wavelength.}

\item \textbf{Describe at least three methods to probe the gravitational potential of galaxies, their assumptions, and their realm of applicability.}

\end{enumerate}

\section*{High Energy}

\begin{enumerate}

\item \textbf{Derive, approximately or exactly up to dimensionless factors, the Bondi accretion rate onto a mass $M$ at rest in a homogeneous medium of density $\rho$ and speed of sound $c_s$. Explain the significance of the Bondi radius.}

Bondi accretion is the spherical accretion of interacting particles. We're in the non-relativistic, steady state limit. 

Begin with the Bernoulli Equation:

\begin{equation}
\frac{1}{2} u^2 + \int \frac{dP}{\rho} - \frac{GM}{r} = e = \mathrm{constant}
\end{equation}

Far away from the black hole, you can neglect the velocity $u$ because it goes as $\frac{1}{r^4}$. Out there, the conserved quantity is just the second two terms. You can write the first in terms of the sound speed:

$$ \int \frac{dP}{\rho} = \frac{1}{\gamma-1} c_s^2 $$

since $c_s^2 = \frac{\gamma P}{\rho}$. 

Thus, we have the conservation statement

$$ \frac{1}{\gamma-1} c_s^2 - \frac{GM}{r} = \frac{1}{\gamma-1} c_{s,\infty}^2 $$

The Bondi radius is where the sound speed has changed by order unity from its original value. It defines the radius of influence of the black hole. And, it marks the transition from subsonic (far from the BH) to supersonic (close to the BH). 

At supersonic speeds, 

\begin{equation}
r_\mathrm{Bondi} = \frac{GM}{c^2_\infty}
\end{equation}

At this transition point, the mass accretion rate is governed by conservation of mass:

$$ \dot{M} = 4 \pi r^2 \rho u = 4 \pi \rho u \left(\frac{GM}{c^2_\infty}\right)^2 $$

With some fudge factor $\lambda$,

\begin{equation}
\dot{M} = \lambda 4 \pi \left(\frac{GM}{c^2_\infty}\right)^2 \rho_\infty c_\infty
\end{equation}

\item \textbf{Discuss the difference between high-mass X-ray binaries and low mass X-ray binaries (both containing neutron stars). What are the typical spin period and magnetic fields of the neutron stars in the two types of system, and how are these measured or estimated? How are the spin period and magnetic fields changing with time?}

In many X-ray binary systems, the spin period is observed to be decreasing steadily on a timescale of around 10,000 years. Presumably, this is caused by torques induced by the accretion process. Observed spin periods are usually between 1 second and 1000 seconds. 

Observe LMXB (accretion from Roche lobe overflow, orbital period hours) with coherent X-ray pulsing. The X-ray pulsing continues even when the X-ray intensity drops by large factors. Surface magnetic field of around $10^9$\,G, far lower than the fields inferred in pulsing sources in HMXB (accretion from the stellar wind of a massive (O or B) star, orbital period $\sim$months, $B \sim 10^{12}$\,G). May suggest that the fields decay, perhaps as a result of accretion. You can also measure weak magnetic fields from quasi-periodic oscillations detected in the power spectra of LMXBs. Spin period on the order of milliseconds. These are the progenitors of the majority of millisecond pulsars. 

HMXB systems emit x-ray pulsations, LMXB systems emit x-ray bursts.

\item \textbf{Derive the equation for the effective temperature of an
accretion disk around a black hole of mass $M$ with accretion rate
$M_\odot$ as a function of radius $r$. Specify the assumptions
required to get an answer, and comment on what could go wrong with them.
Define the Eddington luminosity and explain its relevance to the peak
frequency of the emitted radiation.}

Consider a fully ionized gas and assume it's unmagnetized. There's some flux coming in and on average, photons come out perpendicular. The Thompson cross section is

\begin{equation}
\sigma_T = \frac{8 \pi}{3} \left( \frac{e^2}{m_e c^2} \right)^2
\end{equation}

So the radiation force is

\begin{equation}
\mathrm{Force} = \sigma_T \frac{F}{c}
\end{equation}

Now, assume that for each electron, radiation force balances gravity ($m_e g$) and solve for the luminosity limit for keeping electrons bound:

$$ L_e < \frac{4 \pi G M m_e c}{\sigma_T} = 7 \times 10^{34} \left( \frac{M}{M_\odot} \right) \, \mathrm{erg/s}$$

There are lots of stars, white dwarfs, and neutron stars that have much higher luminosity than this. The mistake is that we assumed that only the gravitational force was acting, but there's also a (stronger) electric force:

$$ \frac{e Q}{r} > \frac{G M m_e} {r^2} $$

When this is the case, you can estimate the \emph{fraction} of electrons that are lost by the number of electrons lost over the total number of electrons: $\frac{Q/e}{M/m_p}$. You can write this in terms of the ratio of proton mass to electron mass and the fine structure constant. When you do, you find that the number is \emph{tiny}: the moral is that there aren't that many electrons to be lost. 

Instead of individual electrons, let's consider ions: you have to remove an entire ion with its electrons in order to unbind. The balance between gravitational force and radiation force becomes

$$ M_{\mathrm{ion}} g = F_{\mathrm{rad, ion}} $$

Since the electron mass is negligible compared to the proton mass, $M_{\mathrm{ion}} \sim A m_u$. $F_{\mathrm{rad, ion}}$ is just the sum of the radiation force on each electron, so $F_{\mathrm{rad, ion}} = Z F_{\mathrm{rad, e}}$. The maximum luminosity the star can have without losing ions to radiation is therefore: 

$$ A m_u \frac{GM}{r^2} > \frac{L}{4 \pi r^2 c} Z \sigma_T $$

This gives us our definition of the \textbf{Eddington luminosity}:

\begin{equation}
\ledd \equiv \frac{4 \pi G M c}{Z \sigma_T / A m_u} = \frac{4 \pi G M c}{\sigma_T (1 + X) / (2 m_u)}
\end{equation}

where $X$ is the hydrogen mass fraction. 

To estimate a numerical value for this, put it in terms of the $L_e$ that we solved for earlier, and the $m_p/m_e$ ratio. This gives:

$$ \ledd = 1.3 \times 10^{38} \frac{2}{1+X} \left( \frac{M}{M_\odot} \right) \mathrm{erg/s} $$

This could be violated under the following circumstances:

\begin{enumerate}
\item Different opacity: dust or photoionization (this is why red giants have enormous dusty winds) or strong magnetic fields (quantum electrodynamic effects).
\item Electrons are actually not bound; instead, they're flowing out due to a wind (radiation convection).
\end{enumerate}

When you accrete, you basically drop stuff onto a hard surface. The energy released is the change in potential energy $ \frac{G M \dot{M} }{R} $. In order to accrete, you must be radiating below $\ledd$; otherwise, the effective gravity would be in the other direction. So, $\ledd$ sets the maximum luminosity above which the object can no longer accrete. This therefore corresponds to a maximum mass accretion rate:

\begin{equation}
\medd \equiv \frac{4 \pi c R}{\sigma_T (1 + X) / 2 m_u}
\end{equation}

Numerically,

$$ \medd = 1.5 \times 10^{-8} \, \frac{M_\odot}{\mathrm{yr}} \left( \frac{R}{10\,\mathrm{km}} \right) \left( \frac{2}{1+X} \right) $$

\textcolor{red}{In short, radiation pressure on electrons balances with gravity on nuclei. $\mathrm{Force}=\mathrm{Flux} \times \bar{Z}\sigma_{T}/c=GM \bar{A}m_u/r^2$. For disk accretion, radiation is not necessarily to be isotropic. }

\end{enumerate}

\section*{ISM}

\begin{enumerate}

\item \textbf{Discuss the principal heating mechanisms and cooling
transitions which determine the temperature of the principal phases
of the interstellar medium: cold molecular, warm neutral, warm
ionized, and hot ionized.}

\item \textbf{Explain how the temperature, density, ionizing spectrum
and element abundances of an HII region can be estimated using
measurements of different line transitions.}

\item \textbf{Derive the dimensional scaling of the Jeans' length for
gravitational instability (remember you can do this either based on
formal stability analysis, or to order of magnitude from physical arguments.) Discuss its relevance to star formation.}

\end{enumerate}

\section*{Extragalactic/Cosmology}

\begin{enumerate}

\item \textbf{Put on a timeline, and describe the principal events in the
thermal history of the universe, from $kT=10\,TeV$ to $kT = 0.1\,eV$.}

\item \textbf{Give a semi-quantitative discussion of the connection between fluctuations of the CMB on angular scales of arcminutes to degrees, and the baryonic structure (galaxies, clusters, correlations of galaxies) observed in the local universe, redshift $z < 0.5$.}

\item \textbf{Which elements/isotopes are produced in BBN and in what quantities? Explain qualitatively how the yield of each depends on the cosmic baryon density and why.}

\end{enumerate}

\end{document}


