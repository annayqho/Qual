% Simple article, but with some useful ApJ things included
% Created by AYQH, 18 June 2016
\documentclass[12pt, letterpaper, preprint]{aastex}
\bibliographystyle{apj}
\usepackage{float, bm, graphicx, subfigure, amsmath, morefloats}
\usepackage{color}
\usepackage{tabularx}

\setlength\parindent{0pt}

% useful macros
\newcommand{\ledd}{L_{\mathrm{Edd}}}
\newcommand{\medd}{\dot{M}_{\mathrm{Edd}}}
\newcommand{\Comment}[2]{ [{\color{red}\sc #1 :} {{\color{cyan} \it #2}}]}
\newcommand{\gcm}{g/$\mathrm{cm}^3$}
\newcommand{\angstrom}{\mbox{\AA}}
\newcommand{\vel}{\textbf{v}}
\newcommand{\bfield}{\textbf{B}}
\newcommand{\acc}{\textbf{a}}

\begin{document}

\title{2015 ``Will Ask'' Qual Questions}

\section*{Radiative Processes}

\begin{enumerate}

  \item \textbf{Derive the total power and characteristic frequency 
      of synchrotron radiation from a relativistic particle of mass $m$, 
      charge $e$, and energy $E$ moving in a magnetic field $B$. 
      Use this to explain why synchrotron radiation is generally 
      negligible for protons.}

The total power is

\begin{equation}
  P = \frac{4}{3} \, \sigma_T \, c \, \gamma^2 \, \beta^2 \, u_B
  \label{eq:relativistic-larmor}
\end{equation}

where $\sigma_T$ is the Thomson cross section $\frac{8}{3} \pi r_0^2$,
$u_B$ is the magnetic energy density $\frac{B}{8 \pi}$.

The characteristic frequency of gyration is

\begin{equation}
\omega_g = \frac{eB}{\gamma mc}
\end{equation}

To derive this, begin with the equation of motion resulting from the Lorentz force:

\begin{equation}
  \frac{d}{dt} (\gamma m \vel) = \frac{e}{c} \, (\vel \times \mathbf{B})
  \label{eq:lorentz-force}
\end{equation}

For a particle with constant energy $E = \gamma m c^2$,
we can take $\gamma$ to be constant.
Therefore, Equation \ref{eq:lorentz-force} can be rewritten

\begin{equation}
  \frac{d \vel}{dt} = \frac{e}{\gamma m c} \, (\vel \times \mathbf{B})
  \label{eq:lorentz-force-rearranged}
\end{equation}

Because of the cross product in Equation \ref{eq:lorentz-force-rearranged}, the magnetic field only changes the component of the velocity that is perpendicular to it. Thus, Equation \ref{eq:lorentz-force-rearranged} can be rewritten

\begin{equation}
  \acc = \frac{d \vel_\bot}{dt} = \frac{e}{\gamma m c} \, (\vel_\bot \times \mathbf{B})
  \label{eq:lorentz-force-perp}
\end{equation}

Because $\vel_\parallel$ and $\vel_\mathrm{tot}$ are constant,
$\vel_\bot$ must be constant as well. Assuming an unchanging magnetic field, the acceleration in Equation \ref{eq:lorentz-force-perp} is constant. We thus have uniform circular motion in the ($\vel_\bot$, \bfield) plane. Together with the constant velocity normal to this plane, we have helical motion.

Because the motion is circular, the acceleration in Equation \ref{eq:lorentz-force-perp} can be taken to be a centripetal acceleration
with magnitude

\begin{equation}
a = v_\bot \omega_g = \frac{e v_\bot B}{\gamma m c}
\label{eq:acc}
\end{equation}

So, the corresponding characteristic frequency of rotation is

\begin{equation}
\omega_g = \frac{e B}{\gamma m c}
\end{equation}

To calculate the power, we use the \textbf{relativistic Larmor formula}
which gives the power radiated by a particle moving at relativistic speeds. It is derived via Lorentz transformations, taking the non-relativistic Larmor formula to hold true in the frame of the particle.
We use only one term of it, since the acceleration is perpendicular to the velocity.

\begin{equation}
  P = \frac{2}{3} \frac{\gamma^4 e^2 a^2}{c^3}
  \label{rel-larmor-formula}
\end{equation}

Now we plug in our expression for acceleration in Equation \ref{eq:acc} and average the formula over all angles (assuming an isotropic distribution of velocities). If $\alpha$ is the angle between the velocity of the particle and the magnetic field, 

\begin{equation}
\langle v_\bot^2 \rangle = \frac{v^2}{4 \pi} \int \sin^2 \alpha \, d\Omega
= \frac{2}{3} v^2
\end{equation}

With this averaging and some substitutions,

\begin{equation}
  P = \frac{4}{3} \, \sigma_T \, c \, \gamma^2 \, \beta_\bot^2 \, u_B
  \label{relativistic-larmor}
\end{equation}

The timescale on which power is lost due to sycnchrotron radiation
is about $\frac{E}{dE/dt}$. Since $E \sim \gamma m_0 c^2$ and
$ P = dE/dt$, we find that the timescale scales as $\gamma$. 
$\gamma$ itself scales as $E/E_0$ where $E_0$ is the rest mass energy
of the particle. This is roughly 0.511\,MeV for an electron and 938\,MeV
for a proton. So, for a given energy, the timescale is much longer
for a proton than it is for an electron.

Also, note that $\omega_g$ is not actually the frequency of radiation.
In practice, the frequency spectrum is complex and can extend up to many times $\omega_g$. Important effects include: 

\begin{itemize}

\item 
Broad spectrum due to relativistic beaming: the particle's radiation is beamed within $1/\gamma$ of a cone with half-angle equal to the pitch angle. This cone is the characteristic angular distribution of radiation emitted by a particle with perpendicular acceleration and velocity. Thus, instead of a sinusoidal light curve, you get a series of sharp pulses. Recall that the Fourier Transform of a short time pulse is a broad frequency spectrum with $\Delta \omega \sim 1/T$.

\item
The time between pulses is a factor of $\gamma^3$ narrower than what you'd get from $\omega_g$ due to the Doppler effect. 

\item
The single-particle spectrum extends to
something of order $\omega_c$ before falling away,
where $\omega_c \sim \gamma^3 \omega_g$. 

\item
The spectrum of a distribution of a population of relativistic electrons can be approximated by a power law over a limited range of frequencies. For a power law distribution of particle energies with index $p$ (that is, $N(E) dE = C E^{-p} dE$) over a sufficiently broad energy range, the spectral index of the radiation is $s = (p-1) / 2$. 

\end{itemize}

  \item \textbf{Explain the connection between detailed balance, 
    the Einstein A and B relations,
    Kirchoff's law and the Milne relations, 
    and give an example of their use to connect the
    bremsstrahlung emission spectrum and the free-free absorption coefficient.}
    
\textbf{Detailed balance relations} govern the relationship between inverse processes on an atomic scale.
These processes could be transitions between atomic states
(the \textbf{Einstein A and B relations})
or photoionization and recombination (the \textbf{Milne relations}).
These atomic relationships are indifferent to whether the system is in thermal equilibrium: they don't care about the distribution of the intensity of radiation or the velocities of particles.
And yet, they are intimately connected to macroscopic-level relationships that \emph{do} depend on thermal conditions.
For example, the Einstein A and B relations
are implied by the macroscopic relationship between
emission and absorption (\textbf{Kirchoff's Law}).
Similarly, the \textbf{Bremsstrahlung emission spectrum} 
is intimately connected with absorption on an atomic scale characterized by the \textbf{free-free absorption coefficient}.

Here are the relations in more detail.

To understand the \textbf{Einstein A and B relations},
consider a system with two discrete energy levels 
$L_1$ and $L_2$ with statistical weights $g_1$ and $g_2$ respectively. To move between $L_1$ and $L_2$, the system
can do three things: emit a photon in the absence of a radiation field (spontaneous emission), absorb a photon (absorption),
or emit a photon in the presence of a radiation field (stimulated emission). The transition probability per unit time of these three processes are atomic properties, characterized
by the Einstein coefficients $A_{21}$, $B_{12}$, and $B_{21}$ respectively. The relations between these coefficients are: 

\begin{equation}
  g_1 B_{12} = g_2 B_{21}
  \label{einstein-B}
\end{equation}

\begin{equation}
  \frac{A_{21}}{B_{21}} = \frac{2 h \nu^3}{c^2}
  \label{einstein-A}
\end{equation}

To derive these relations, one invokes three consequences of thermodynamic equilibrium. 
The first is that the number of transitions into an atomic state is equal to the number of transitions out of that state:

\begin{equation}
  n_1 B_{12} \bar{J} = n_2 B_{21} \bar{J} + n_2 A_{21}
  \label{detailed-balance}
\end{equation}

where $\bar{J}$ is the intensity of the radiation field
weighted by the importance of different frequencies
in causing transitions (in practice, ``weighted" means
integrated over the line profile function):

\begin{equation}
  \bar{J} \equiv \int_0^\infty J_\nu \, \phi(\nu) \, d\nu
  \label{jbar}
\end{equation}

The second is that the relative number densities in the two levels are governed by the Boltzmann Equation:

\begin{equation}
  \frac{n_1}{n_2} = \frac{g_1}{g_2} \, e^{h \nu / k_B T}
  \label{boltzmann-eq}
\end{equation}

And the third is that $J_\nu = B_\nu$ and, further, since
$B_\nu$ varies slowly on $\Delta \nu$, that $\bar{J} = B_\nu$. 

Note that even though we invoke thermal equilibrium to derive these relations, the relations themselves have no dependence on temperature and thus hold regardless of whether the atoms are in thermal equilibrium. They are an extension of the law governing matter in thermodynamic equilibrium on a macroscopic scale, known as...

\textbf{Kirchoff's law for thermal equilibrium}
says that a blob of material at thermal equilibrium
(a blob whose emission depends solely on its temperature
and internal properties)
has a source function that obeys a blackbody spectrum.
In other words, the emitting and absorbing properties
of such a blob encourage radiation to approach
the Planck function $B_\nu(T)$.
More formally, this law relates
the material's absorption coefficient $\alpha_\nu$
to its emission coefficient $j_\nu$
and its temperature $T$:

\begin{equation}
  j_\nu = \alpha_\nu B_\nu (T)
  \label{kirchoff-law}
\end{equation}

This says that a good emitter is a good absorber, in
thermal equilibrium.

The \textbf{Milne Relations} govern the relationship between
photoionization and its inverse, radiative recombination.
In radiative recombination, an ion captures an electron
into a bound state $n$ and emits a photon. 
Just like we did for the Einstein relations
we use four consequences of thermal equilibrium to
derive a result that is independent of thermal equilibrium. 

The first consequence
is that the velocities of the electrons obey
a thermal distribution:

\begin{equation}
f(v) = 4 \pi \left( \frac{m}{2 \pi kT} \right)^{3/2} v^2 e^{-mv^2 / 2 kT}
\end{equation}

The second consequence is that the intensity of the radiation field is the Planck function $B_\nu (T)$.

The third consequence is that the relative number densities of ions $N_+$, electrons $N_e$, and neutron atoms $N_n$ are governed by the Saha equation:

\begin{equation}
\frac{N_+ N_e}{N_n} = \left( \frac{2 \pi m k T}{h^2} \right)^{3/2} \frac{g_e g_+}{g_n} e^{-\chi / kT} 
\end{equation}

Finally, the number of recombinations per time per volume due to thermal electrons is equal to the number of photoionizations per time per volume for a blackbody radiation field:

\begin{equation}
N_+ N_e \sigma_{fb} f(v) v dv = 
\frac{4 \pi}{h \nu} N_n \sigma_{bf} (1-e^{-h\nu/kT}) B_\nu d\nu
\end{equation}

where $\sigma_{fb} (v)$ is the cross-section for recombination
and $\sigma_{bf} (v)$ is the cross-section for photoionization.

Using these four properties together, we obtain the Milne relation:

\begin{equation}
\frac{\sigma_{bf}}{\sigma_{fb}} = \left(\frac{mcv}{h\nu} \right)^2 \frac{g_e g_+}{2 g_n}
\end{equation}


Now, let's turn to the application of these relations
to connecting \textbf{the Bremsstrahlung emission spectrum}
to \textbf{the free-free absorption coefficient}.
\textbf{Bremsstrahlung}, or \textbf{free-free emission},
is radiation due to the acceleration of a charge in the Coulomb field of another charge. For example, an electron under the influence of an ion's Coulomb field accelerates and radiates; the ion is much more massive than the electron and therefore its acceleration and radiation are negligible and for practical purposes its Coulomb field can be considered fixed.

You need quantum mechanics to fully understand this process, because the energies of the photons produced are comparable to that of the particle being accelerated. However, in some regimes you can justify a classical treatment to achieve the correct functional dependence, then make up for leaving out quantum mechanics by tacking on corrections called Gaunt factors. 

One such regime is the small-angle scattering regime, in which the electron's path is basically straight because the electron is moving so rapidly. In this regime, you can calculate the change in the electron's energy over the course of the time it spends closely interacting with the ion. In thermal conditions, you can integrate the resulting emission spectrum over a thermal distribution of particle speeds. Here is the resulting emission spectrum for \emph{thermal Bremsstrahlung}:

\begin{equation}
  \frac{dW}{dt dV dv} = \epsilon_\nu^{ff} = 4 \pi j_\nu^{ff} 
  = 6.8 \times 10^{-38} \, T^{-1/2} Z^2 n_e n_i \, e^{h \nu / kT}
  \, \bar{g}_{ff} (\nu,T)
  \label{eq:bremsstrahlung}
\end{equation}

Now, we do what we did before: relate a macroscopic process to an atomic process. In this case, the atomic process is thermal free-free absorption, the absorption of radiation by an electron moving in the field of an ion. Kirchoff's law tells us that

$$ j_\nu^{ff} = \alpha_\nu^{ff} B_\nu (T) $$

Using Equation \ref{eq:bremsstrahlung} we can solve for $\alpha_\nu^{ff}$, the free-free absorption coefficient:

\begin{equation}
\alpha_\nu^{ff} = 3.7 \times 10^8 \, T^{-1/2} Z^2 n_e n_i \nu^{-3} \, (1-e^{-h\nu/kT}) \, \bar{g}_{ff}
\end{equation}

In the Wien limit, $\alpha_\nu^{ff} \propto \nu^{-3}$.
In the RJ limit, $\alpha_\nu^{ff} \propto \nu^{-2}$.
So, low energy (low frequency) electrons are
preferentially absorbed. 

  \item Draw the energy levels of the hydrogen atom and 
    identify which transitions are allowed. 
    Which ones are in the visible part of the spectrum? 
    Which level has no allowed decays, and what is its main decay mode?
\end{enumerate}

\section*{Instrumentation}

\begin{enumerate}
  \item \textbf{Describe quantitatively the point spread function 
      of a diffraction-limited optical telescope. 
      Explain how diffraction spikes arise, 
      and what determines their position and intensities. 
      Under what circumstances will the PSF be broadened by atmospheric
      turbulence?}
\end{enumerate}

\section*{Stars}
\begin{enumerate}
  \item \textbf{Explain what the Hayashi track is, 
      and describe what types of objects live on it.
      Qualitatively explain how it arises and what 
      assumptions are required for its derivation.}
  \item \textbf{Estimate the temperature as a function 
      of depth in the Sun’s convection zone.
      What is the temperature at the base of the convection zone?}
\end{enumerate}

\section*{Galaxies}

\begin{enumerate}

\item \textbf{Define two-body relaxation and estimate its time scale in (a) a globular cluster, (b) the Milky Way's disk. In which cases is two-body relaxation important?}

\item \textbf{Draw qualitatively the spectral energy distribution of the Milky Way, and describe how its morphology might appear to an external observer as a function of wavelength.}

\item \textbf{Describe at least three methods to probe the gravitational potential of galaxies, their assumptions, and their realm of applicability.}

\end{enumerate}

\section*{High Energy}

\begin{enumerate}

\item \textbf{Derive, approximately or exactly up to dimensionless factors, the Bondi accretion rate onto a mass $M$ at rest in a homogeneous medium of density $\rho$ and speed of sound $c_s$. Explain the significance of the Bondi radius.}

Bondi accretion is the spherical accretion of interacting particles. We're in the non-relativistic, steady state limit. 

Begin with the Bernoulli Equation:

\begin{equation}
\frac{1}{2} u^2 + \int \frac{dP}{\rho} - \frac{GM}{r} = e = \mathrm{constant}
\end{equation}

Far away from the black hole, you can neglect the velocity $u$ because it goes as $\frac{1}{r^4}$. Out there, the conserved quantity is just the second two terms. You can write the first in terms of the sound speed:

$$ \int \frac{dP}{\rho} = \frac{1}{\gamma-1} c_s^2 $$

since $c_s^2 = \frac{\gamma P}{\rho}$. 

Thus, we have the conservation statement

$$ \frac{1}{\gamma-1} c_s^2 - \frac{GM}{r} = \frac{1}{\gamma-1} c_{s,\infty}^2 $$

The Bondi radius is where the sound speed has changed by order unity from its original value. It defines the radius of influence of the black hole. And, it marks the transition from subsonic (far from the BH) to supersonic (close to the BH). 

At supersonic speeds, 

\begin{equation}
r_\mathrm{Bondi} = \frac{GM}{c^2_\infty}
\end{equation}

At this transition point, the mass accretion rate is governed by conservation of mass:

$$ \dot{M} = 4 \pi r^2 \rho u = 4 \pi \rho u \left(\frac{GM}{c^2_\infty}\right)^2 $$

With some fudge factor $\lambda$,

\begin{equation}
\dot{M} = \lambda 4 \pi \left(\frac{GM}{c^2_\infty}\right)^2 \rho_\infty c_\infty
\end{equation}

\item \textbf{Discuss the difference between high-mass X-ray binaries and low mass X-ray binaries (both containing neutron stars). What are the typical spin period and magnetic fields of the neutron stars in the two types of system, and how are these measured or estimated? How are the spin period and magnetic fields changing with time?}

In many X-ray binary systems, the spin period is observed to be decreasing steadily on a timescale of around 10,000 years. Presumably, this is caused by torques induced by the accretion process. Observed spin periods are usually between 1 second and 1000 seconds. 

Observe LMXB (accretion from Roche lobe overflow, orbital period hours) with coherent X-ray pulsing. The X-ray pulsing continues even when the X-ray intensity drops by large factors. Surface magnetic field of around $10^9$\,G, far lower than the fields inferred in pulsing sources in HMXB (accretion from the stellar wind of a massive (O or B) star, orbital period $\sim$months, $B \sim 10^{12}$\,G). May suggest that the fields decay, perhaps as a result of accretion. You can also measure weak magnetic fields from quasi-periodic oscillations detected in the power spectra of LMXBs. Spin period on the order of milliseconds. These are the progenitors of the majority of millisecond pulsars. 

HMXB systems emit x-ray pulsations, LMXB systems emit x-ray bursts.

\item \textbf{Derive the equation for the effective temperature of an
accretion disk around a black hole of mass $M$ with accretion rate
$M_\odot$ as a function of radius $r$. Specify the assumptions
required to get an answer, and comment on what could go wrong with them.
Define the Eddington luminosity and explain its relevance to the peak
frequency of the emitted radiation.}

Consider a fully ionized gas and assume it's unmagnetized. There's some flux coming in and on average, photons come out perpendicular. The Thompson cross section is

\begin{equation}
\sigma_T = \frac{8 \pi}{3} \left( \frac{e^2}{m_e c^2} \right)^2
\end{equation}

So the radiation force is

\begin{equation}
\mathrm{Force} = \sigma_T \frac{F}{c}
\end{equation}

Now, assume that for each electron, radiation force balances gravity ($m_e g$) and solve for the luminosity limit for keeping electrons bound:

$$ L_e < \frac{4 \pi G M m_e c}{\sigma_T} = 7 \times 10^{34} \left( \frac{M}{M_\odot} \right) \, \mathrm{erg/s}$$

There are lots of stars, white dwarfs, and neutron stars that have much higher luminosity than this. The mistake is that we assumed that only the gravitational force was acting, but there's also a (stronger) electric force:

$$ \frac{e Q}{r} > \frac{G M m_e} {r^2} $$

When this is the case, you can estimate the \emph{fraction} of electrons that are lost by the number of electrons lost over the total number of electrons: $\frac{Q/e}{M/m_p}$. You can write this in terms of the ratio of proton mass to electron mass and the fine structure constant. When you do, you find that the number is \emph{tiny}: the moral is that there aren't that many electrons to be lost. 

Instead of individual electrons, let's consider ions: you have to remove an entire ion with its electrons in order to unbind. The balance between gravitational force and radiation force becomes

$$ M_{\mathrm{ion}} g = F_{\mathrm{rad, ion}} $$

Since the electron mass is negligible compared to the proton mass, $M_{\mathrm{ion}} \sim A m_u$. $F_{\mathrm{rad, ion}}$ is just the sum of the radiation force on each electron, so $F_{\mathrm{rad, ion}} = Z F_{\mathrm{rad, e}}$. The maximum luminosity the star can have without losing ions to radiation is therefore: 

$$ A m_u \frac{GM}{r^2} > \frac{L}{4 \pi r^2 c} Z \sigma_T $$

This gives us our definition of the \textbf{Eddington luminosity}:

\begin{equation}
\ledd \equiv \frac{4 \pi G M c}{Z \sigma_T / A m_u} = \frac{4 \pi G M c}{\sigma_T (1 + X) / (2 m_u)}
\end{equation}

where $X$ is the hydrogen mass fraction. 

To estimate a numerical value for this, put it in terms of the $L_e$ that we solved for earlier, and the $m_p/m_e$ ratio. This gives:

$$ \ledd = 1.3 \times 10^{38} \frac{2}{1+X} \left( \frac{M}{M_\odot} \right) \mathrm{erg/s} $$

This could be violated under the following circumstances:

\begin{enumerate}
\item Different opacity: dust or photoionization (this is why red giants have enormous dusty winds) or strong magnetic fields (quantum electrodynamic effects).
\item Electrons are actually not bound; instead, they're flowing out due to a wind (radiation convection).
\end{enumerate}

When you accrete, you basically drop stuff onto a hard surface. The energy released is the change in potential energy $ \frac{G M \dot{M} }{R} $. In order to accrete, you must be radiating below $\ledd$; otherwise, the effective gravity would be in the other direction. So, $\ledd$ sets the maximum luminosity above which the object can no longer accrete. This therefore corresponds to a maximum mass accretion rate:

\begin{equation}
\medd \equiv \frac{4 \pi c R}{\sigma_T (1 + X) / 2 m_u}
\end{equation}

Numerically,

$$ \medd = 1.5 \times 10^{-8} \, \frac{M_\odot}{\mathrm{yr}} \left( \frac{R}{10\,\mathrm{km}} \right) \left( \frac{2}{1+X} \right) $$

\textcolor{red}{In short, radiation pressure on electrons balances with gravity on nuclei. $\mathrm{Force}=\mathrm{Flux} \times \bar{Z}\sigma_{T}/c=GM \bar{A}m_u/r^2$. For disk accretion, radiation is not necessarily to be isotropic. }

\end{enumerate}

\section*{ISM}

\begin{enumerate}

\item \textbf{Discuss the principal heating mechanisms and cooling
transitions which determine the temperature of the principal phases
of the interstellar medium: cold molecular, warm neutral, warm
ionized, and hot ionized.}

\item \textbf{Explain how the temperature, density, ionizing spectrum
and element abundances of an HII region can be estimated using
measurements of different line transitions.}

\item \textbf{Derive the dimensional scaling of the Jeans' length for
gravitational instability (remember you can do this either based on
formal stability analysis, or to order of magnitude from physical arguments.) Discuss its relevance to star formation.}

\end{enumerate}

\section*{Extragalactic/Cosmology}

\begin{enumerate}

\item \textbf{Put on a timeline, and describe the principal events in the
thermal history of the universe, from $kT=10\,TeV$ to $kT = 0.1\,eV$.}

\item \textbf{Give a semi-quantitative discussion of the connection between fluctuations of the CMB on angular scales of arcminutes to degrees, and the baryonic structure (galaxies, clusters, correlations of galaxies) observed in the local universe, redshift $z < 0.5$.}

\item \textbf{Which elements/isotopes are produced in BBN and in what quantities? Explain qualitatively how the yield of each depends on the cosmic baryon density and why.}

\end{enumerate}

\end{document}


